% PLEASE USE THIS FILE AS A TEMPLATE
% Check file iosart2c.tex for more examples
%
% Journal:
%   Journal of Ambient Intelligence and Smart Environments (jaise)
%   Web Intelligence and Agent Systems: An International Journal (wias)
%   Semantic Web: Interoperability, Usability, Applicability (SW)
% IOS Press
% Latex 2e

% options: jaise|wias|sw
% add. options: [seceqn,secfloat,secthm,crcready,onecolumn]


\documentclass{iosart2c}

%\documentclass[sw]{iosart2c}
%\documentclass[wias]{iosart2c}
%\documentclass[jaise]{iosart2c}

\usepackage[T1]{fontenc}
\usepackage{times}%
\usepackage{natbib}% for bibliography sorting/compressing
%\usepackage{amsmath}
%\usepackage{endnotes}
\usepackage{graphics}

%%%%%%%%%%% Put your definitions here
\usepackage[utf8]{inputenc}
\usepackage[hyphens]{url}
\usepackage{verbatim} 

%% Define a new 'smallurl' style for the package that will use a smaller font.
\makeatletter
\def\url@smallurlstyle{%
  \@ifundefined{selectfont}{\def\UrlFont{\sf}}{\def\UrlFont{\scriptsize\ttfamily}}}
\makeatother
%% Now actually use the newly defined style.
\urlstyle{smallurl}
\newcommand{\nofootnote}[1]{~#1}

%%%%%%%%%%% End of definitions

\pubyear{0000}
\volume{0}
\firstpage{1}
\lastpage{1}

\begin{document}

\begin{frontmatter}

%\pretitle{}
\title{What'chu talkin' about, Willis?\\A Comparison of Conversation Topics\\on Facebook and Twitter}
% \footnote{\url{http://en.wikipedia.org/wiki/Diff'rent\_Strokes\#Later\_appearances\_of\_the\_characters}}
\runningtitle{A Comparison of Conversation Topics on Facebook and Twitter}
%\subtitle{}

%\review{}{}{}


% For one author:
%\author{\fnms{} \snm{}\thanks{}}
%\address{sdds}
%\runningauthor{}

% Two or more authors:
\author[A]{\fnms{Thomas} \snm{Steiner}\thanks{T. Steiner is partially supported by the European Commission under Grant No. 248296 FP7 I-SEARCH project}},
\author[B]{\fnms{Arnaud} \snm{Brousseau}},
\author[C]{\fnms{Raphaël} \snm{Troncy}},
\author[D]{\fnms{Ruben} \snm{Verborgh}},
\author[E]{\fnms{Rik} \snm{Van de Walle}},
\author[F]{\fnms{Joaquim} \snm{Gabarró Vallés}}
\runningauthor{}
\address[A]{Google Germany GmbH, ABC-Str. 19, 20354 Hamburg, Germany,\\
E-mail: tomac@google.com}
\address[B]{Google Germany GmbH, ABC-Str. 19, 20354 Hamburg, Germany,\\ 
E-mail: arnaud.brousseau@gmail.com}
\address[C]{EURECOM, Sophia Antipolis, France\\
E-mail: raphael.troncy@eurecom.fr}

\begin{abstract}
The Twitter Trends feature allows for a global or local view on ``what's happening in my world right now" from a tweet producers' point of view. In this paper, we show the possibility to complete the functionality provided by Twitter Trends via having a closer look at the other side: the tweet consumers' -- i.e., readers' -- point of view. While Twitter Trends works by analyzing the frequency of terms and their velocity of appearance in tweets being written, our approach is based on the popularity of extracted named entities (in the sense of Linked Data) in tweets being read. Our experimentation architecture uses a client-side browser extension to harvest and dissect tweets from users' timelines, search result pages, or profile pages, i.e., tweets supposed to be read. Named entities are extracted via several third-party Natural Language Processing (NLP) Web services in parallel, and are then reported to Google Analytics, which is used to store, analyze, and compute trends by pivoting the reported named entities by Google Analytics data, e.g., users' geographic locations.
\end{abstract}

\begin{keyword}
 \sep
\end{keyword}

\end{frontmatter}

%%%%%%%%%%% The article body starts:

\section{Introduction}

%%%%%%%%%%% The bibliography starts:
\bibliography{swj2012}

\end{document}
