\documentclass{letter}
\usepackage[a4paper,margin=4cm]{geometry}
\usepackage[english]{babel}
\usepackage[T1]{fontenc}
\usepackage[utf8]{inputenc}
\usepackage[tracking,kerning,spacing]{microtype}
\usepackage{times}
\usepackage[pdftex,urlcolor=black,colorlinks=true,linkcolor=black,citecolor=black]{hyperref} % PDF hyperlinks + autoref command

\begin{document}
{\huge A consumer's look on Facebook and Twitter}

{\LARGE Reply to the review comments}

\vspace{5em}

Dear Aba-Sah Dadzie, Matthew Rowe, and Millan Stankovic,

We thank you for the comments we have received on our submission \emph{A consumer's look on Facebook and Twitter~-- What do people read and where?} that was initially submitted to the Special Issue on The Semantics of Microposts (\url{http://www.semantic-web-journal.net/blog/special-issue-semantics-microposts}).
\\*
The enclosed paper is a~throughly rewritten version of our original submission,
addressing the issues and concerns you have put forward.

One of the major improvements concerns a better justification of the claimed benefits of analyzing from the reader perspective instead of the traditional author viewpoint.
In each of the reported events, we have indicated the contribution of our reader-focused method.

Secondly, we have vastly improved the Related Work section to better address the expectations of the Semantic Web Journal audience.
The cited works cover a broader range, and the differences between the approaches have been emphasized.

We have also addressed issues regarding user privacy when reader-oriented methods are put into practice.
Other changes with regards to this concern are clarifications on how the information was retrieved.

Finally, Guiseppe Rizzo has joined the author team, adding an in-depth overview of named entity extraction and elaborating on various points in this regard.

We thank you beforehand for considering the inclusion of our paper in the Semantic Web Journal.

Best regards,

Thomas Steiner\\*
Ruben Verborgh\\*
Raphaël Troncy\\*
Guiseppe Rizzo\\*
Rik Van de Walle\\*
Joaquim Gabarro\\*
Arnaud Brousseau

\end{document}